\documentclass[a4paper,12pt]{article}
\usepackage[MeX]{polski}
\usepackage[utf8]{inputenc}
\usepackage{graphicx}

%opening
\title{Toyota Celica}
\author{Jarema Rydzwski-Bączek}

\begin{document}


\maketitle


\begin{abstract}
,,Toyota Celica'' --- sportowy samochód osobowy produkowany przez japońską firmę Toyota w latach 1970-2005. Dostępny był jako liftback i coupé, w niewielkiej liczbie egzemplarzy powstawała też wersja kabriolet. Do napędu używano benzynowych silników R4 o pojemności 1,6-2,4 l. W latach 1970-1986 Celica zaliczała się do samochodów tylnonapędowych, od 1986 do końca produkcji - przednionapędowych. W latach 1986-1999 powstawała topowa wersja Celiki z napędem na cztery koła oraz turbodoładowaniem - GT-Four. Powstało siedem generacji modelu, 2. i 3. posłużyły jako baza dla stworzenia nowego samochodu - Toyoty Supry. Nazwa pochodzi od hiszpańskiego słowa celica.
\end{abstract}
\section{I Generacja}

Pierwsza generacja Celiki trafiła na rynek japoński pod koniec 1970 roku. Zaprojektowana została jako bardziej przystępna alternatywa dla droższej Toyoty 2000GT. Nowy model dzielił płytę podłogową z modelem Carina, dostępny był jako 3-drzwiowy liftback lub 2-drzwiowe coupé. Premiera miała miejsce podczas Tokyo Motor Show w październiku 1970 roku. Produkcja wersji coupé ruszyła w grudniu tego samego roku, liftback trafił zaś do sprzedaży w kwietniu 1973. Był to pierwszy samochód typu specialty car (odpowiednik amerykańskiego pony cars w Japonii).

Do napędu wersji coupé nabywca mógł wybierać spośród czterech jednostek benzynowych, trzy 1600 (2T - 100 KM, 2T-B - 105 KM i 2T-G - 115 KM) i jedna 1400 (T - 86 KM). Napęd przenoszony był na oś tylną poprzez 4 --- lub 5-biegową manualną bądź 3-biegową automatyczną skrzynię biegów. Dostępne były cztery wersje wyposażenia: GT, ST, LT i ET. W 1975 model przeszedł facelifting.

Liftback wprowadzony w kwietniu 1973 opierał się na prototypie SV-1 zaprezentowanym na Tokio Motor Show w 1971 roku[5]. Pojazd był 3-drzwiowym liftbackiem z układem miejsc 2+2 z nadwoziem stylizowanym na fastbacka. Do napędu służyły m.in. silniki 2000 18R-G (145 KM) i 1600 2T-G (115 KM).


\begin{figure}
\includegraphics[width=0.5\hsize] {toyota.png}
\caption{Toyota celica}\label{fig:toyota}
\end{figure}



\begin{table}
\begin{tabular}{lc}
\hline
\textbf{Wersje}&\textbf{Silnik}\\
\hline
1,4 T & R4 1{,}4 l (1407 cm$^3$)\\
1,6 2T & R4 1{,}6 l (1589 cm$^3$)\\
2,0 18 R & R4 2{,}0 l (1968 cm$^3$)\\
\hline
\end{tabular}
\end{table}



\end{document}